\documentclass[UTF8]{ctexart}
\title{汽车构造拆装实习报告}
\author{1852479 康嘉梁}
\date{}
\ctexset{
bibname = {参考文献},
}
\pagestyle{plain}

\usepackage[a4paper]{geometry}
\geometry{left=3.18cm,right=3.18cm,top=2.54cm,bottom=2.54cm}

\usepackage{amsmath}
\numberwithin{figure}{section}
\numberwithin{table}{section}

% \renewcommand\thesection{\chinese{section}}
% \renewcommand\thesubsection{\arabic{subsection}}
% \renewcommand\thesubsubsection{\thesubsection\thinspace.\thinspace\arabic{subsubsection}}

\usepackage[version=4]{mhchem}

\usepackage{siunitx}

\usepackage{cleveref}
\crefname{section}{节}{节}
\crefname{equation}{式}{式}
\crefname{figure}{图}{图}
\crefname{table}{表}{表}
\crefname{appendix}{附录}{附录}
\newcommand{\crefpairconjunction}{~和~}
\newcommand{\crefmiddleconjunction}{、}
\newcommand{\creflastconjunction}{~和~}
\newcommand{\crefpairgroupconjunction}{~和~}
\newcommand{\crefmiddlegroupconjunction}{、}
\newcommand{\creflastgroupconjunction}{~和~}
\newcommand{\crefrangeconjunction}{~}

\usepackage{graphicx}
\DeclareGraphicsExtensions{.pdf,.eps,.jpg,.png}
\graphicspath{{img/body}, {img/gearbox}, {img/automobile electronics}}
\usepackage{caption}
\captionsetup[figure]{labelsep=period}
\usepackage{float} %设置图片浮动位置的宏包
\usepackage{subfigure} %插入多图时用子图显示的宏包

\begin{document}
\maketitle

\section{车身部分}
\subsection{请简述从发动机排气歧管到车辆排气口,由哪些主要部分组成。各部分的主要功能是什么?}

\label{subsection:1.1}

对于实验室中的领克02来说,从发动机排气歧管排出的尾气经排气管、涡轮机(\cref{turbine})、三元催化转换器及颗粒捕集器(\cref{three-way catalytic converter})、前消声器(\cref{muffler F})、后消声器(\cref{muffler B}),最后由左右两个排气尾管(\cref{tail pipe})排到大气中。

\begin{figure}[htbp]
	\centering
	\begin{minipage}[b]{0.3\textwidth}
		\centering
		\includegraphics[width=\textwidth]{turbine}
		\caption{涡轮机}
		\label{turbine}
	\end{minipage}
	\begin{minipage}[b]{0.4\textwidth}
		\centering
		\includegraphics[width=\textwidth]{three-way catalytic converter}
		\caption{三元催化转换器}
		\label{three-way catalytic converter}
	\end{minipage}
	\begin{minipage}[b]{0.25\textwidth}
		\centering
		\includegraphics[width=\textwidth]{muffler F}
		\caption{前消声器}
		\label{muffler F}
	\end{minipage}
	\begin{minipage}[b]{0.55\textwidth}
		\centering
		\includegraphics[width=\textwidth]{muffler B}
		\caption{后消声器}
		\label{muffler B}
	\end{minipage}
	\begin{minipage}[b]{0.4\textwidth}
		\centering
		\begin{minipage}[b]{0.45\textwidth}
			\centering
			\includegraphics[width=\textwidth]{tail pipe L}
		\end{minipage}
		\begin{minipage}[b]{0.45\textwidth}
			\centering
			\includegraphics[width=\textwidth]{tail pipe R}
		\end{minipage}
		\caption{排气尾管}
		\label{tail pipe}
	\end{minipage}
\end{figure}

涡轮机是废气涡轮增压器(\cref{exhaust turbocharger})的一个重要组成部分,它的作用是把发动机排出废气的能量转化为机械功来驱动压气机叶轮。本车上使用的径流式涡轮,废气经进气蜗壳由叶轮的径向流入,经轴向流出。涡轮叶片一般在\SI{900}{\celsius}的高温排气冲击下工作,承受巨大的离心力,由镍基耐热合金钢或陶瓷材料制成;蜗壳用耐热合金铸铁铸成,外表面呈黑色,内表面光洁。

由于增压器的热负荷较大,单靠机油不足以满足增压器散热需求,\cref{exhaust turbocharger detached}中间那根较长的软管便是冷却水的管路,与发动机的冷却系统联通。冷却液自增压器中间体上的冷却液进口流入中间体水套,从冷却液出口流回发动机冷却系统。

\cref{exhaust turbocharger detached}所示的涡轮增压系统还带有放气阀(\cref{wastgate valve})。当增压压力达到一阈值后,ECU控制放气阀打开使部分排气不经过涡轮直接排出增压器,防止增压压力和涡轮转速过大造成压气机的阻塞。

\begin{figure}[htbp]
	\centering
	\begin{minipage}[b]{0.4\textwidth}
		\centering
		\includegraphics[width=\textwidth]{exhaust turbocharger}
		\caption{废气涡轮增压器}
		\label{exhaust turbocharger}
	\end{minipage}
	\centering
	\begin{minipage}[b]{0.35\textwidth}
		\centering
		\includegraphics[width=0.8\textwidth]{exhaust turbocharger detached}
		\caption{拆下的废气涡轮增压器}
		\label{exhaust turbocharger detached}
	\end{minipage}
	\begin{minipage}[b]{0.18\textwidth}
		\centering
		\includegraphics[width=\textwidth]{wastgate valve}
		\caption{放气阀}
		\label{wastgate valve}
	\end{minipage}
\end{figure}

三元催化转换器以排气中的\ce{CO}和\ce{HC}作为还原剂,在把\ce{NO_x}还原成\ce{N2}和\ce{O2}的同时把\ce{CO}和\ce{HC}氧化为\ce{CO2}和\ce{H2O},反应式为
\begin{align}
	\ce{CO + O2 & -> CO2} \nonumber       \\
	\ce{HC + O2 & -> H2O + CO2} \nonumber \\
	\ce{NO_X    & -> N2 + O2} \nonumber
\end{align}
三元催化转换器将带有很多小孔的蜂窝状陶瓷作为载体,表面附着一层薄的氧化铝中间镀层,再在其上镀以由铂、钯、铑等贵金属制成的催化剂。催化转换器外面用金属外壳封闭,并焊在排气管路内,避免被随意拆卸。三元催化转换器要求车辆使用无铅汽油,温度超过\SI{350}{\celsius}时才会工作,最重要的是要求发动机混合气浓度始终在理论空燃比附近。最后这点是通过氧传感器实现的。

三元催化转换器前后各有一个氧传感器,其中前氧传感器起主要作用,后氧传感器的用处主要是检测催化转换器是否工作正常。ECU通过进气量、喷油量及排气中氧浓度可实现过量空气系数$\lambda$的闭环控制。氧传感器的传感元件一般是在陶瓷管基体上附着有\ce{ZrO2}涂层,外侧通排气,内侧通大气。当陶瓷管温度在\SI{350}{\celsius}左右时具有固态电解质的特性。如\cref{voltage response curve},当混合气偏浓时,排气中\ce{O2}浓度低,陶瓷管内外\ce{O^2-}浓度差较高,氧传感器输出电压高(约\SI{900}{\mV});反之当混合气偏稀时输出电压较低(约\SI{100}{\mV})。氧传感器输出电压在理论空燃比附近对过量空气系数的变化很敏感,ECU据此不断修正喷油量使得$\lambda$始终在最佳范围内。

由于缸内直喷易产生碳烟,排气系统上设置有颗粒捕集器以满足排放法规要求。

\begin{figure}[htbp]
	\centering
	\begin{minipage}[b]{0.45\textwidth}
		\centering
		\includegraphics[width=\textwidth]{voltage response curve}
		\caption{氧传感器的电压曲线}
		\label{voltage response curve}
	\end{minipage}
\end{figure}

消声器的主要功能是减小排气的噪声。由于发动机为间歇排气,在排气管中会引起压力的脉动,排气的温度、压力、速度等也很高,具有一定能量。消声器的通过吸收、扩张、共振、干涉等方式耗散排气中的能量,达到消声的目的。一个消声器中常有不同形式的多个腔组合(\cref{muffler dissect}),以提高消声效果。

\begin{figure}[htbp]
	\centering
	\begin{minipage}[b]{0.6\textwidth}
		\centering
		\includegraphics[width=\textwidth]{muffler dissect}
		\caption{剖开的排气消声器}
		\label{muffler dissect}
	\end{minipage}
\end{figure}

\subsection{请简述汽车动力总成如何固定在车身上。}

发动机和变速箱刚性连接构成汽车动力总成,但动力总成与车身的连接是柔性的,这需要通过悬置来实现。悬置的功用主要有支撑、限位和隔振。支撑指悬置必须能承受动力总成的质量,使其不至于产生过大的静位移;限位指动力总成在受到各种干扰力作用的情况下,悬置应能有效的限制其最大位移以避免干涉;隔振指悬置一方面要阻止作为振源的发动机向车身传递振动,另一方面还要能缓和路面不平激励等传给发动机的振动和冲击。

从隔振角度来说,希望悬置较软,以期将振动更好地隔离;而从支承和限位角度来说,为使发动机舱能布置紧凑,又希望悬置越硬越好。在悬置设计中如何最优化选取悬置刚度是一个极为重要的问题。同时,为了使振动得到迅速衰减,发动机悬置还应具有适当的的阻尼。

领克02车型的动力总成采用三点固定的方式,通过一个后悬置和两个前悬置将动力总成与车身连接,如\cref{mount}。\cref{mount B}看得较清楚,我们可以看到,后悬置与发动机和车身的两个连接轴线互相垂直,且两连接处都有橡胶减震块,能有效衰减动力总成相对车身俯仰和偏航两个方向的相对运动,而对侧倾的衰减主要由前面两悬置实现。

当然,这种软连接的形式会对发动机的动力输出造成一定影响,因为部分扭矩被用于克服减震块的阻力。在一些超跑和方程式赛车中有将发动机直接固定在单体壳上的方式,以期最大化动力输出,但会恶化车辆的NVH (Noise, vibration, and harshness) 特性,在乘用车上一般都是用三点式或四点式的悬置将汽车动力总成固定在车身上。

\begin{figure}[htbp]
	\centering
	\subfigure[后悬置]{\includegraphics[width=0.6\textwidth]{mount B}\label{mount B}}
	\subfigure[右前悬置]{\includegraphics[width=0.4\textwidth]{mount RF}\label{mount RF}}
	\subfigure[左前悬置]{\includegraphics[width=0.5\textwidth]{mount LF}\label{mount LF}}
	\caption{悬置}
	\label{mount}
\end{figure}

\subsection{请简述发动机的控制系统中的传感器和执行器主要有哪些?}

发动机控制系统中的传感器主要有空气流量计、曲轴位置传感器、氧传感器、节气门位置传感器、冷却液温度传感器、爆震传感器、离合器开关;执行器主要有燃油泵继电器、燃油泵、喷油器、点火线圈、凸轮轴调节阀、节气门电机(\cref{sensors and actuators})。

\begin{figure}[htbp]
	\centering
	\includegraphics[width=0.9\textwidth]{sensors and actuators}
	\caption{发动机的控制系统中的传感器和执行器}
	\label{sensors and actuators}
\end{figure}

\cref{sensors}所示为发动机控制系统中的部分传感器。现在车用的空气流量计多能直接测得空气的质量流量,分为热线式和热膜式两种,ECU根据空气供给的多少确定喷油量,保证空燃比在理论空燃比附近。曲轴位置传感器有磁电式、光电式和霍尔式等,与安装在曲轴上的靶轮共同工作,主要用于用检测发动机转速,并在凸轮轴相位传感器的配合下确定各缸压缩上止点。前氧传感器和后氧传感器结构类似,都能将排气中氧浓度和空气氧浓度的差转化为输出电压供ECU实现$\lambda$闭环控制。节气门位置传感器是电子节气门系统的一部分,让ECU能闭环控制节气门位置。冷却液温度传感器装在发动机冷却液的出水口上,用来检测出水温度,这个温度一般控制在\SI{90}{\celsius}左右,过高或过低对发动机的工作都是不利的。冷却液温度可通过节温器进行反馈调节,当温度较低时只让冷却液在小回路中循环,便于发动机快速升温;温度高时开启大回路,给发动机较快降温。爆震传感器安装在发动机机体上,它的固有频率设计为与爆震发生时机体的震动频率相近。当发生爆震时,发动机机体与传感器发生共振,传感器输出最大电压信号,表示爆震发生,ECU据此推迟点火,即减小点火提前角,这能有效抑制发动机爆燃。

\begin{figure}[htbp]
	\centering
	\subfigure[空气流量计]{\includegraphics[width=0.45\textwidth]{airflowing sensor}\label{airflowing sensor}}
	\subfigure[前氧传感器]{\includegraphics[width=0.45\textwidth]{oxygen sensor F}\label{oxygen sensor F}}
	\subfigure[后氧传感器]{\includegraphics[width=0.45\textwidth]{oxygen sensor B}\label{oxygen sensor B}}
	\subfigure[冷却液温度传感器]{\includegraphics[width=0.45\textwidth]{coolant temperature sensor}\label{coolant temperature sensor}}
	\caption{发动机的控制系统中的部分传感器}
	\label{sensors}
\end{figure}

燃油泵继电器、燃油泵、喷油器、油箱等都是燃油供给系统(\cref{fuel-supplying system})的组成部分。对于缸内直喷汽油机,供油压力都很高,达到\qtyrange[range-phrase = $\,\sim\,$, range-units = single]{300}{450}{\kPa},需采用高压油轨,油轨中压力在发动机运转时变化不大,ECU对喷油器的控制主要是控制其喷油时刻和喷油持续时间。燃油泵继电器只在起动时和发动机运转时接通电动燃油泵,提供足够的燃油压力和富余燃油;当发动机发生事故而停止运转时,燃油泵也停止运转,降低起火风险。油箱内因汽油蒸发产生的高压蒸气通到炭罐(\cref{charcoal canister})中,炭罐底部接大气,上部一边接油箱一边接进气管,ECU在需要时开启炭罐控制阀,燃油蒸气在压力差的作用下被导入到进气管中,随后进入气缸中燃烧。当然,炭罐系统的存在也使得燃油供给量不完全由喷油量控制,且经过炭罐的蒸气量不便计量,这也是需要设置氧传感器进行$\lambda$闭环控制的原因之一。

点火系统要能保证按时并提供足量能量点燃燃烧室内混合气。最佳点火提前角受很多因素影响,其中最主要的是发动机转速和混合气的燃烧速度,而混合气的燃烧速度又和混合气的成分、发动机的结构(包括燃烧室形状和压缩比)等有关。当发动机转速一定时,随着负荷(节气门开度)的增加,进入气缸的可燃混合气增多,压缩终了时的压力和温度升高,同时残余废气在缸内混合气中所占的比例下降,混合气燃烧速度增大,点火提前角应适当减小;反之负荷减小时点火提前角应加大。当节气门开度一定时,发动机转速升高,燃烧过程所占曲轴转角增大,应适当加大点火提前角,否则燃烧会延续到做功冲程使得功率和经济性下降。对于抗爆性较好的汽油,许用的点火提前角较大,加注不同牌号的汽油后点火提前角也应相应调节。ECU确定点火提前角后,发出信号令点火系统点火。电控点火系统分为有分电器式和无分电器式两种,现在用的多为无分电器点火系统,又称直接点火系统。直接点火系统又有单火花点火线圈、双火花点火线圈和四火花点火线圈等,\cref{ignition system}所示为双火花点火线圈,1、4缸和2、3缸分别同时点火,这样点火时一个气缸在压缩冲程的上止点附近,另一个气缸运行到排气冲程上止点附近,但只有一个缸中有可燃混合气,可以被点燃。

\cref{VVT}所示的是一种张紧轮式进气相位调节装置,排气凸轮相位固定,液压缸使链条的上(下)部分张紧且长度发生变化,从而带动进气凸轮相对旋转一角度实现可变气门正时。怠速时,进气门延迟关闭,改善燃烧稳定性;中、低转速时,进气门提前关闭,以获得较大转矩;高转速时,进气门延迟关闭,利用进气惯性提高气缸充气量以提高功率。

在电子节气门系统(\cref{electronic throttle})中,节气门与加速踏板间无拉索的机械连接,节气门开度完全受ECU控制。当发动机不运转且点火开关打开时,ECU只根据加速踏板位置控制节气门控制器。当发动机运转时,ECU对节气门的控制不完全依赖加速踏板的输入,譬如说可在加速踏板仅踏下一半时即令节气门全开,减少节流损失。有电子节气门的发动机还不需怠速调节器,进气量可精确控制,在提升车辆驾驶性能的同时降低排放和油耗。

二次空气系统(\cref{secondary air})和排气再循环系统(\cref{EGR})都是发动机排放控制系统的一部分,现代发动机管理系统将喷油控制和排放控制作为统一的整体,以满足日趋严格的排放法规。二次空气系统主要在发动机冷起动和暖机过程中工作,在每缸排气门后面导入空气,使得高温废气中的\ce{HC}和\ce{CO}在排气管中继续燃烧,产生热量让催化转换器尽快升温。排气再循环系统通过回引部分废气与新鲜空气一起参与燃烧,降低混合气中\ce{O2}含量和燃烧温度从而抑制\ce{NO_x}的产生。

正如\cref{subsection:1.1}所提到的那样,废气涡轮增压系统(\cref{turbocharging})上的放气阀也受ECU控制。

\begin{figure}[htbp]
	\centering
	\subfigure[燃油供给系统]{\includegraphics[width=0.3\textwidth]{fuel-supplying system}\label{fuel-supplying system}}
	\subfigure[炭罐]{\includegraphics[width=0.3\textwidth]{charcoal canister}\label{charcoal canister}}
	\subfigure[点火系统]{\includegraphics[width=0.3\textwidth]{ignition system}\label{ignition system}}
	\subfigure[可变气门正时]{\includegraphics[width=0.3\textwidth]{VVT}\label{VVT}}
	\subfigure[电子油门]{\includegraphics[width=0.3\textwidth]{electronic throttle}\label{electronic throttle}}
	\subfigure[二次空气]{\includegraphics[width=0.3\textwidth]{secondary air}\label{secondary air}}
	\subfigure[排气再循环]{\includegraphics[width=0.3\textwidth]{EGR}\label{EGR}}
	\subfigure[废气涡轮增压]{\includegraphics[width=0.3\textwidth]{turbocharging}\label{turbocharging}}
	\caption{发动机的控制系统中的部分执行器}
	\label{actuators}
\end{figure}

\subsection{请简述车身的主要功能及其实现方式。}

汽车车身是驾驶人的工作场所,也是装载乘客和货物的场所。

车身为驾驶员提供良好的操作条件,为乘员提供舒适的乘坐条件,并保证完好无损地运载货物且装卸方便。车身四周装有车窗,车身内外配备后视镜,部分新款汽车在车身周围还安装了多组传感器,驾驶员能清晰地感知车周环境。车内有仪表板,座椅、方向盘、后视镜等附件的位置可调,方便驾驶员掌握车辆情况。对乘员来说,发动机舱盖上有隔音棉(\cref{sound insulation cotton}),车门和车顶中空除方便车窗、天窗开闭外也可提升隔音、隔热效果,炽热的排气管与车身底部间覆盖隔热材料(\cref{insulation}),可开合的门四周覆盖胶条密封以隔音、防水,舒适的座椅起部分隔振功能,这些措施都让乘员与外界较恶劣的环境隔绝,提升舒适性。车身上开门位置合理,乘员能方便上下车,货物能方便装卸。对于货车来说车身上设有货物固定装置,避免运输时货物滑移、倾倒。

\begin{figure}[htbp]
	\centering
	\begin{minipage}[b]{0.7\textwidth}
		\centering
		\includegraphics[width=\textwidth]{sound insulation cotton}
		\caption{隔音棉}
		\label{sound insulation cotton}
	\end{minipage}
	\begin{minipage}[b]{0.25\textwidth}
		\centering
		\includegraphics[width=\textwidth]{insulation}
		\caption{隔热材料}
		\label{insulation}
	\end{minipage}
\end{figure}

碰撞安全性是车身结构和设备设计时需要重点考虑的。\cref{BIW}为白车身的一部分,注意到涂有红色油漆的钢梁均由高强钢制成,起防撞梁的作用。保险杠安装梁与车身本体间有缓冲装置,部分高配车型上这里会安装弹性元件和阻尼器,缓和正碰时冲击。车门内部也有高强度的防撞梁,增强侧碰安全性。车上装有多个安全气囊,和安全带一起避免碰撞时乘员重要部位直接与硬物接触,保护乘员。现在对行人保护日趋重视,车身上通过吸能保险杠、软性的引擎盖材料、大灯及附件无锐角等措施保护低速碰撞时行人的安全。

\begin{figure}[htbp]
	\centering
	\begin{minipage}[b]{0.6\textwidth}
		\centering
		\includegraphics[width=\textwidth]{BIW}
		\caption{白车身}
		\label{BIW}
	\end{minipage}
\end{figure}

车身外形的设计要考虑空气动力学特性,以便汽车行驶时能有效引导周围的气流,提高汽车动力性和行驶稳定性,并改善发动机舱冷却条件和车内通风。注意到展示的领克02上发动机底板已被拆下(\cref{engine compartment bottom}),这能让我们方便地看到油底壳等发动机部件,但在车辆行驶过程中会带来较大的风阻增加,故实际车辆上发动机舱底部有一平板用于引导气流,如\cref{engine compartment with floor}所示,该底板还可兼起保护油底壳免受异物碰撞的作用。对于一般的发动机前置车型,车身头部开有进气孔,还可布置冷排,供发动机进气、冷却。现在的车身越来越向流线型发展,这主要也是出于降阻的考虑。

\begin{figure}[htbp]
	\centering
	\begin{minipage}[b]{0.4\textwidth}
		\centering
		\includegraphics[width=\textwidth]{engine compartment bottom}
		\caption{发动机舱底部}
		\label{engine compartment bottom}
	\end{minipage}
	\begin{minipage}[b]{0.5\textwidth}
		\centering
		\includegraphics[width=\textwidth]{engine compartment with floor}
		\caption{带底板的发动机舱}
		\label{engine compartment with floor}
	\end{minipage}
\end{figure}

车身是一件精致的艺术品,以其优雅的雕塑形体、装饰件和内部覆饰以及悦目的色彩使人获得美感享受,反映时代的风貌、民族的传统和独特的企业形象。每辆车前后都有车标,但很多时候我们看清车标前就已能知道车辆的品牌。宝马的前脸、普桑的尾灯、法拉利的赛车红、劳斯莱斯的欢庆女神,凡此种种都已成为一款车家族性的标志。

现在乘用车常用承载式车身,车身还得承担起车架的功能,包括支撑连接汽车的各零部件、并承受来自车内外的各种载荷。车身既要承力,又要尽可能地轻量化,对现代的车身设计提出了更高要求。

% \subsection{请简述自动变速箱的换挡离合器和换挡制动器的主要组成部分及其工作原理。}
% \subsection{请简述增压发动机进气系统的主要组成部分及其功能。}
\clearpage

\section{变速箱部分}
\subsection{手动变速器一般有哪些档位? 试述各档的动力传递路线。}

\label{subsection:2.1}

手动变速器一般为普通齿轮式变速器,也称轴线固定式变速器,它按照变速器传动齿轮轴的数目,可分为两轴式变速器和三轴式变速器(也称中间轴式变速器)。目前,轿车和轻、中型货车变速器通常有\numrange[range-phrase = $\,\sim\,$]{3}{5}个前进挡和一个倒挡,重型货车用组合式变速器中会有更多的挡位。三轴式变速器有真正的直接挡,部分变速器还设传动比小于1的超速挡,供良好路面行驶用。一般我们说的变速器挡位数,都指的是前进挡位数。

\cref{two-shaft manual transmission}为一用于纵置发动机的两轴式变速器(注意到图上注为2/3挡同步器的实为3/4挡同步器),和我们在拆装实践中用的那一型结构十分相似,动力传动路线也已在图中标出。对于该变速器:\\
1挡的动力传递路线为:输入轴(包括其上的1挡齿轮)$\rightarrow$输出轴1挡齿轮$\rightarrow$接合齿圈$\rightarrow$接合套$\rightarrow$花键毂$\rightarrow$输出轴;\\
2挡的动力传递路线为:输入轴(包括其上的2挡齿轮)$\rightarrow$输出轴2挡齿轮$\rightarrow$接合齿圈$\rightarrow$接合套$\rightarrow$花键毂$\rightarrow$输出轴;\\
3挡的动力传递路线为:输入轴$\rightarrow$花键毂$\rightarrow$接合套$\rightarrow$接合齿圈$\rightarrow$输入轴3挡齿轮$\rightarrow$输出轴3挡齿轮$\rightarrow$输出轴;\\
4挡的动力传递路线为:输入轴$\rightarrow$花键毂$\rightarrow$接合套$\rightarrow$接合齿圈$\rightarrow$输入轴4挡齿轮$\rightarrow$输出轴4挡齿轮$\rightarrow$输出轴;\\
5挡的动力传递路线为:输入轴$\rightarrow$花键毂$\rightarrow$接合套$\rightarrow$接合齿圈$\rightarrow$输入轴5挡齿轮$\rightarrow$输出轴5挡齿轮$\rightarrow$输出轴;\\
倒挡的动力传递路线为:输入轴$\rightarrow$输入轴倒挡齿轮$\rightarrow$倒挡中间齿轮$\rightarrow$输出轴倒挡齿轮$\rightarrow$输出轴。

\begin{figure}[htbp]
	\centering
	\begin{minipage}[b]{\textwidth}
		\centering
		\includegraphics[width=\textwidth]{two-shaft manual transmission}
		\caption{两轴式5挡手动变速器}
		\label{two-shaft manual transmission}
	\end{minipage}
\end{figure}

\cref{three-shaft manual transmission}为课本上提到的一种三轴式变速器,它的特点是具有直接挡。在该变速器中,1、倒挡采用接合套换挡,其余各挡位采用同步器换挡,其中2挡使用锁销式同步器以承受更大载荷,\numrange[range-phrase = $\,\sim\,$]{3}{6}挡使用锁环式同步器。对该变速器:\\
1挡的动力传递路线为:第一轴$\rightarrow$第一轴常啮齿轮$\rightarrow$中间轴常啮齿轮$\rightarrow$中间轴$\rightarrow$中间轴1挡齿轮$\rightarrow$第二轴1挡齿轮$\rightarrow$接合齿圈$\rightarrow$接合套$\rightarrow$花键毂$\rightarrow$第二轴;\\
2挡的动力传递路线为:第一轴$\rightarrow$第一轴常啮齿轮$\rightarrow$中间轴常啮齿轮$\rightarrow$中间轴$\rightarrow$中间轴2挡齿轮$\rightarrow$第二轴2挡齿轮$\rightarrow$接合齿圈$\rightarrow$接合套$\rightarrow$花键毂$\rightarrow$第二轴;\\
3挡的动力传递路线为:第一轴$\rightarrow$第一轴常啮齿轮$\rightarrow$中间轴常啮齿轮$\rightarrow$中间轴$\rightarrow$中间轴3挡齿轮$\rightarrow$第二轴3挡齿轮$\rightarrow$接合齿圈$\rightarrow$接合套$\rightarrow$花键毂$\rightarrow$第二轴;\\
4挡的动力传递路线为:第一轴$\rightarrow$第一轴常啮齿轮$\rightarrow$中间轴常啮齿轮$\rightarrow$中间轴$\rightarrow$中间轴4挡齿轮$\rightarrow$第二轴4挡齿轮$\rightarrow$接合齿圈$\rightarrow$接合套$\rightarrow$花键毂$\rightarrow$第二轴;\\
5挡的动力传递路线为:第一轴$\rightarrow$第一轴常啮齿轮$\rightarrow$中间轴常啮齿轮$\rightarrow$中间轴$\rightarrow$中间轴5挡齿轮$\rightarrow$第二轴5挡齿轮$\rightarrow$接合齿圈$\rightarrow$接合套$\rightarrow$花键毂$\rightarrow$第二轴;\\
6挡(直接挡)的动力传递路线为:第一轴$\rightarrow$第一轴常啮齿轮$\rightarrow$接合齿圈$\rightarrow$接合套$\rightarrow$花键毂$\rightarrow$第二轴;\\
倒挡的动力传递路线为:第一轴$\rightarrow$第一轴常啮齿轮$\rightarrow$中间轴常啮齿轮$\rightarrow$中间轴$\rightarrow$中间轴倒挡齿轮$\rightarrow$倒挡中间齿轮$\rightarrow$第二轴倒挡齿轮$\rightarrow$接合齿圈$\rightarrow$接合套$\rightarrow$花键毂$\rightarrow$第二轴。

\begin{figure}[htbp]
	\centering
	\begin{minipage}[b]{\textwidth}
		\centering
		\includegraphics[width=\textwidth]{three-shaft manual transmission}
		\caption{三轴式6挡手动变速器}
		\label{three-shaft manual transmission}
	\end{minipage}
\end{figure}

\subsection{试述自锁与互锁装置的特点和作用。}

为保证变速器在任何情况下都能准确、安全、可靠地工作,其操纵机构必须设置安全装置,包括自锁、互锁和倒挡锁装置。对于六挡变速器,还应设置选挡锁装置。

自锁装置如\cref{self-locking}所示,它的作用主要有三。一是保证操纵变速杆推动拨叉前、后移动距离足够,使得滑动齿轮(或接合套)与相应的齿轮(或接合齿圈)在全齿宽上啮合,提高齿轮的寿命和承载力。二是保证在全齿宽啮合后汽车遇到较大颠簸或其它情况时,滑动齿轮(或接合套)不会自动产生轴向移动造成啮合长度的减小甚至是脱挡。三是为驾驶员提供良好的换挡手感,在钢球滑入凹槽后给驾驶员明显的反馈提示他换挡已到位。

自锁装置由自锁弹簧和自锁钢球组成,每根拨叉轴的表面有与钢球对应的凹槽(\cref{self-locking slots})。当任意一根拨叉轴连同拨叉一起轴向移动到空挡或某一工作挡位时,必有一个凹槽正对钢球,于是钢球被弹簧压嵌入该凹槽内,拨叉轴的轴向位置即被固定,从而拨叉连同滑动齿轮(或接合套)也被固定在空挡或一工作挡位上。当需要换挡时,驾驶员必须通过变速杆对拨叉和拨叉轴施加一定的轴力,这个力又在凹槽与拨叉轴接触的侧面上产生一克服弹簧弹力的纵向分力,将自锁钢球压回孔中,拨叉轴方可轴向移动。拨叉轴表面相邻自锁凹槽间的距离就等于使得在全齿宽上啮合或完全退出啮合所需的拨叉及拨叉轴的轴向移动距离。

注意到,该型变速器的1、2挡拨叉和3、4挡拨叉上均有三个自锁凹槽(\cref{self-locking slots}),中间的一个凹槽较浅,对应空挡,两边的凹槽较深,对应工作挡位。空挡的自锁凹槽较浅的原因可能是在空挡时与拨叉轴连接的滑动齿轮(或接合套)不参与动力传递,可能引起跳挡的干扰力小,且若某个拨叉轴已挂入工作挡位,其它拨叉轴的空挡位置即被互锁保证而无需过强的自锁。略感意外的是,5、倒挡拨叉轴上仅有两自锁凹槽,其中一个较浅,推测为空挡自锁,另一较深的应为5挡自锁。相信倒挡拨叉上有专供倒挡使用的自锁装置,不必在拨叉轴上实现倒挡自锁。

\begin{figure}[htbp]
	\centering
	\begin{minipage}[b]{0.6\textwidth}
		\centering
		\includegraphics[width=\textwidth]{self-locking}
		\caption{自锁装置示意图}
		\label{self-locking}
	\end{minipage}
	\centering
	\begin{minipage}[b]{0.35\textwidth}
		\centering
		\includegraphics[width=\textwidth]{self-locking slots}
		\caption{自锁凹槽}
		\label{self-locking slots}
	\end{minipage}
	\begin{minipage}[b]{0.5\textwidth}
		\centering
		\includegraphics[width=\textwidth]{self-locking slots 5R}
		\caption{5、倒挡拨叉轴上的自锁凹槽}
		\label{self-locking slots 5R}
	\end{minipage}
\end{figure}

互锁装置(\cref{interlocking})的作用就是防止变速器同时换入两个挡位。若变速杆能同时推动两个拨叉,则可同时换入两个挡位,但不同挡位的传动比不同,这样就会造成齿轮间的机械干涉,变速器被卡死,此时若传递的动力不消失,会使传动路线中最薄弱处被破坏,产生严重后果。

互锁装置由互锁钢球和互锁销组成,每根拨叉轴朝向互锁钢球的侧表面上均铣出一个深度相等的凹槽,任一拨叉轴处于空挡位置时其侧面凹槽都恰对准互锁钢球。两互锁钢球直径之和等于相邻两轴表面间距离加一凹槽深度。中间拨叉轴上两互锁凹槽间有一孔将二者连同,孔中置有互锁销,销的长度等于拨叉轴的直径减去一凹槽深。当变速器处于空挡时,互锁凹槽、互锁钢球和互锁销都在同一直线上。当移动中间那根拨叉轴时,轴两侧的内互锁钢球被从凹槽中挤出,推动外互锁钢球分别嵌入外侧两根拨叉轴侧面的互锁凹槽中,将两轴锁止在空挡位置。如此时欲像\cref{interlocking}中右侧子图所示那样移动1、2挡拨叉轴,则须先将中间的3、4挡拨叉轴退回空挡位置,再移动1、2挡拨叉轴,互锁钢球便被从1、2挡拨叉轴上的互锁凹槽挤出,推动互锁销和其它互锁钢球将剩余两拨叉轴锁止于空挡。\cref{interlocking}中推动倒挡拨叉轴的过程也类似。可见,互锁装置结构的设计使得驾驶员用变速杆推动某一拨叉轴时能自动锁止其它所有拨叉轴。

在我们的拆装实践中,事实上互锁销和互锁钢球都已被提前取出,我们只能看到定位互锁钢球的凹槽和通互锁销的小孔。减速器中互锁装置和自锁装置做在一起,都位于拨叉轴的安装孔上。自锁弹簧和自锁钢球组成一部件,拨叉轴被取出后仍能保持在变速器盖中而不脱出;但互锁销和互锁钢球都是分立零件,体积很小,且依赖拨叉轴固定,在拆卸中易被遗失。在自锁和互锁都存在的情况下,拨叉轴当然还是可以安装的,但须保证装入最后一根拨叉轴时,另两轴都位于空挡位置,否则会发生干涉。

\begin{figure}[htbp]
	\centering
	\begin{minipage}[b]{0.65\textwidth}
		\centering
		\includegraphics[width=\textwidth]{interlocking}
		\caption{互锁装置示意图}
		\label{interlocking}
	\end{minipage}
	\begin{minipage}[b]{0.3\textwidth}
		\centering
		\begin{minipage}[b]{0.8\textwidth}
			\centering
			\includegraphics[width=\textwidth]{interlocking slot 34}
		\end{minipage}
		\begin{minipage}[b]{0.8\textwidth}
			\centering
			\includegraphics[width=\textwidth]{interlocking slot 5R}
		\end{minipage}
		\caption{互锁凹槽}
		\label{interlocking slots}
	\end{minipage}
\end{figure}

\subsection{试分析变速箱拆装过程的特点?}

在变速器的拆装过程中,最重要的就是记录下拆卸顺序,并在重新装配时严格按拆卸的倒序安装。

拆卸的顺序可根据各零件间相对位置关系确定。譬如说换挡轴是第一个被拆下的零件,但其上的拨杆会与5挡拨叉上的拨块干涉,需先将拨叉上的定位销取出以拆下拨叉,换挡轴在旋转一角度后便可从轴向抽出。变速器的输入轴(\cref{input shaft})和输出轴(\cref{output shaft})都是过盈装配在变速器壳体上的,拆卸时要借助压床将它们压出,压出时依次给输入、输出轴施加一小位移,并如此循环,直至两轴全被压出,注意避免让圆锥滚子轴承跳出。若一次只压一根轴,可能啮合的斜齿轮间会干涉造成齿轮损坏。输出轴被压出后,其上的1、2挡拨叉轴和拨叉才可被取下。输入轴上的3、4挡齿轮和相应的同步器由轴用挡圈轴向定位,其中固定输入轴4挡齿轮的卡簧上开有两孔,需将卡簧钳两尖头伸入孔中将卡簧撑开后提出,而固定花键毂的卡簧较厚,更好的方法是用专用的卡簧钳摩擦力较大的成平面的两侧面撑开卡簧缺口后取出。卡簧的拆装较为困难,且如果盲目撬开会造成卡簧变形,无法再装上。变速器上用到的三个定位销(\cref{location pins})为弹性圆柱销,中空且母线方向上有一贯穿的缺口。将销钉敲下时,应用磨钝尖端的螺钉抵住销钉边缘,并用锤子敲击螺钉头部。若直接让螺钉深伸入定位销中部的孔用力,可能会撑大定位销,使之难以被取出。

重新装配变速器的过程事实上较拆卸有更多需要注意的点,因为拆卸顺序错误无非是有零件无法拆下,但装配错误可能要到后面试图装其它零件时才能发现,所幸我组没有遇到此类情况。该型变速器上所用同步器的滑块靠两侧的两钢丝弹簧固定,钢丝弹簧开口的一端有一倒钩,另一端是平的。装配同步器时,先将滑块放入花键毂的凹槽中,套上接合套将滑块基本限位住后才能安装钢丝弹簧。安装钢丝弹簧时,要首先让其有钩子的一端钩住滑块背面的凹槽,然后将其余部分压入,这使得钢丝弹簧不会在同步器中周向转动,也不会从轴向脱出。注意到钢丝弹簧只有约3/4圆周长,故而在安装另一面的钢丝弹簧时注意将缺口错开,以免滑块受力不均。我们主要探究的是3、4挡同步器,这个同步器虽向两侧均能工作,分别控制3、4挡,但仔细观察会发现其花键毂的外花键是断开的,外花键两侧不等长,且花键毂的两侧毂与外花键端部的距离不等,总之花键毂并非左右对称。这种设计可能的原因是两挡受力不同,也带来了安装方向的问题。若将同步器翻转过来安装,花键毂能被装上,因为用于定位花键毂的毂翻转后长度不变,但花键毂上外花键端部的位置变了,这就会使得4挡齿轮装不上,又得把同步器整体拆下重装。所以拆卸时就应记下该同步器的方向以便后面的安装。装回输入、输出轴时,自然要和拆下时一样先把两轴摆到齿轮能啮合的位置,且输出轴要带着1、2挡拨叉轴和拨叉一道装回。然而,装回输入、输出轴时压床已不可用,因为轴伸出箱体的长度超出了压床的工作范围。我们采用的办法是将轴的一端顶在台虎钳上,另一侧用套筒顶在放入轴安装孔的轴承上,敲击套筒即可借助相对运动把输入、输出轴安装到位。倒挡轴和倒挡中间齿轮拆下时较易,仅需将倒挡轴敲出,但装入时要确定倒挡齿轮的方向,且因此时输入、输出轴已装上,倒挡齿轮放倒后才能装入,需用螺丝刀将齿轮重新挑到直立位后再装入倒挡轴。由于倒挡轴上带有定位销,装入时销钉要对准壳体上的开槽,且要从倒挡齿轮的另一侧装入,避免定位销和倒挡齿轮干涉。变速器中三个拨叉也都不是左右对称的,需按原方向安装,否则会和自锁干涉。我们的变速器中已拆掉了互锁,故两拨叉轴可同时挂入工作挡位,但为了最后变速杆安装方便和以后的拆卸,我们需将三轴均移至空挡位置。

\begin{figure}[htbp]
	\centering
	\begin{minipage}[b]{0.18\textwidth}
		\centering
		\includegraphics[width=\textwidth]{input shaft}
		\caption{输入轴}
		\label{input shaft}
	\end{minipage}
	\begin{minipage}[b]{0.31\textwidth}
		\centering
		\begin{minipage}[b]{\textwidth}
			\centering
			\includegraphics[width=\textwidth]{gearbox}
			\caption{变速器}
			\label{gearbox}
		\end{minipage}
		\begin{minipage}[b]{\textwidth}
			\centering
			\includegraphics[width=\textwidth]{output shaft}
			\caption{输出轴}
			\label{output shaft}
		\end{minipage}
	\end{minipage}
	\begin{minipage}[b]{0.39\textwidth}
		\centering
		\begin{minipage}[b]{\textwidth}
			\centering
			\includegraphics[width=\textwidth]{synchronous}
			\caption{同步器}
			\label{synchronous}
		\end{minipage}
		\begin{minipage}[b]{\textwidth}
			\centering
			\includegraphics[width=\textwidth]{location pins}
			\caption{定位销}
			\label{location pins}
		\end{minipage}
	\end{minipage}
\end{figure}

\subsection{两轴变速器的工作原理。}

变速器的作用主要表现在三方面:第一,改变传动比,扩大驱动轮的转矩和转速的变化范围;第二,在发动机转向不变的情况下,实现汽车倒退行驶;第三,利用空挡,使得可以中断发动机动力传递,让发动机可以起动、怠速。

两轴式变速器的动力传递主要依靠两个相互平行的轴(输入轴和输出轴)完成,此外还有一根较短的倒挡轴帮助汽车实现倒退行驶。动力从第一轴输入,经一对齿轮传动后,直接由第二轴输出,故其在一般的减速挡时效率高于三轴式变速器,但无法设置高效的真正的直接挡,且输出轴转向和输入轴相反。

两轴式变速器的前进挡一般都采用常啮合斜齿轮,主动齿轮均为右旋,从动齿轮均为左旋,每对啮合斜齿轮中总有一个是空套在轴上。现在变速器的前进挡一般都采用同步器换挡,需换入某一工作挡位时拨动换挡杆,带动换挡滑块和拨叉将接合套与相应挡位空套的那个齿轮上的接合齿圈接合,使得该齿轮与输入轴或输出轴刚性连接,从而传递动力。各挡的动力传递路线已在\cref{subsection:2.1}中提到过。

为实现汽车的倒退行驶,在输入轴的一侧设置有一根较短的倒挡轴。倒挡中间齿轮空套在倒挡轴上(不用滚针轴承),可轴向滑动,空挡时与输入轴和输出轴的倒挡齿轮不在同一平面。我们在拆装实践中发现,该型变速器倒挡采用直齿滑动齿轮换挡,且输入轴倒挡齿轮、输出轴倒挡齿轮和倒挡中间齿轮在换挡时接合的那一侧都做了斜面处理,在换入倒挡时有一定的导向作用。显然换入倒挡前需先停车,故倒挡可不用接合套,拨动倒挡齿轮使其同时与输入、输出轴上两对应齿轮啮合即可。

两轴式变速器的结构简单、紧凑,容易布置,多用于前置前驱(FF)或后置后驱(RR)的普通级和中级轿车上。

\clearpage

\section{汽油机部分}
\subsection{一般的配气机构、进气门间隙和排气门间隙哪个气门间隙更大一些? 为什么?}
\subsection{曲柄连杆机构由哪些零件组成?}
\subsection{试述活塞环的分类与作用。}
\subsection{安装活塞环时应注意什么?}
\subsection{活塞、连杆、连杆盖组合装配时, 应注意什么?}
\subsection{绘制润滑系统或冷却系统原理示意图(二选一) ?}
\clearpage

\section{柴油机部分}
\subsection{请简述柴油机工作原理}
\subsection{请简述柴油机与汽油机之间的区别}
\subsection{请简述柴油机的构造(按照机构逐项说明)}
\clearpage

\section{汽车电子部分}
\subsection{简述 ABS 系统组成及工作原理。}

制动防抱死系统(ABS)主要由轮速传感器、制动压力调节器和电子控制器(ECU)三大部分组成,如\cref{ABS components}所示。

汽车制动时,首先由轮速传感器测出与制动车轮轮速,并将信号送入ECU。ECU中的运算单元计算出车轮速度、滑动率及车轮的加速度,然后由ECU中的控制单元对这些信号加以分析比较,向压力调节器发出制动压力控制指令。压力调节器中的电磁阀直接或间接地控制制动压力的增减,以调节制动器的制动力矩,使之与地面附着状况相适应,防止制动轮抱死。上述过程如\cref{ABS control process}。

ECU中还有故障诊断单元,其作用为对ABS各部件功能进行监测,当有部件工作异常时,通过指示灯或蜂鸣器向驾驶员发出警告,同时令整个ABS系统停止工作,恢复常规制动方式以保证基本的制动功能。

按ABS系统中可以独立进行制动压力调节的管路(通道)数可将ABS分为四通道ABS、三通道ABS、二通道ABS和单通道ABS等。通道数多的对制动压力调节更精细,制动效能更高。

\begin{figure}[htbp]
	\centering
	\begin{minipage}[b]{0.8\textwidth}
		\centering
		\includegraphics[width=\textwidth]{ABS components}
		\caption{ABS系统的组成}
		\label{ABS components}
	\end{minipage}
	\begin{minipage}[b]{0.8\textwidth}
		\centering
		\includegraphics[width=\textwidth]{ABS control process}
		\caption{ABS系统的控制过程}
		\label{ABS control process}
	\end{minipage}
\end{figure}

ABS系统的执行器——液压制动压力调节器(HCU)根据工作原理的不同可分为流通调压式和变容调压式,目前多用流通调压式HCU,它由电磁阀、液压泵和电动机等组成,直接安装在汽车原有的制动管路中,并通过串联在制动主缸和制动轮缸之间的二位二通电磁阀或三位三通电磁阀直接控制轮缸的制动压力增减或保压。\cref{ABS working principle}所示的BOSCH 2S型ABS系统用的就是流通调压式HCU,它的工作过程可分为常规制动、轮缸减压、轮缸保压和轮缸增压四个阶段。

常规制动过程如\cref{ABS working principle a}所示。当车辆速度很小(如小于\SI[per-mode = symbol]{5}{\km\per\hour}或\SI[per-mode = symbol]{8}{\km\per\hour})或制动踏板上施加的力很小时,ABS不介入,电磁阀不通电,阀中柱塞处于图示的最下方,主缸与轮缸的油路联通,通往储液器的流道被关闭,主缸随时直接控制制动油压的增减。

轮缸减压过程如\cref{ABS working principle b}所示。当ECU根据轮速传感器信号判断车轮出现抱死趋势时,即给电磁阀通入较大电流。阀柱塞移至图示的最上方,主缸与轮缸的通路被截断,而轮缸和储液器相接通,轮缸压力下降。与此同时,驱动电机起动,带动液压泵工作,把流回储液器的制动液加压后送入主缸,为下一制动过程做好准备。注意到ABS系统并不会等到抱死时才开始介入减压,因为快要抱死时轮胎与路面摩擦会造成摩擦系数的变换,此时若仅保压也可能发生抱死,必须先减压。

轮缸保压过程如\cref{ABS working principle c}所示。经过若干次减压循环后,轮缸压力下降使得车轮转速回升,ECU判断车轮抱死趋势减小,车轮滑移率进入最佳范围,即给电磁阀通入较小电流,使得阀柱塞下降至图示位置,但并未完全回位,所有油路被截断。此时制动轮缸压力保持不变,以使车轮滑移率尽可能长时间地保持在最佳范围内。

轮缸增压过程如\cref{ABS working principle d}所示。保压过程中,车轮转速可能回升,车轮滑移率逐渐趋近于0,说明此时轮缸提供的制动油压不足,ECU会发送信号使得电磁阀断电,阀柱塞下降到初始位置,主缸与轮缸油路再次接通,主缸的高压制动液重新进入轮缸,使得轮缸中油压回升。若车轮又趋近于抱死,则ABS将继续重复减压——保压——增压的控制循环。

上述的压力调节是脉冲式的,我们分析的ABS 2S系统的控制频率为\qtyrange[range-phrase = $\,\sim\,$, range-units = single]{4}{10}{\Hz},而新式的ABS控制频率可达\qtyrange[range-phrase = $\,\sim\,$, range-units = single]{70}{150}{\Hz}。

\begin{figure}[htbp]
	\centering
	\subfigure[]{\includegraphics[width=0.45\textwidth]{ABS working principle a}\label{ABS working principle a}}
	\subfigure[]{\includegraphics[width=0.45\textwidth]{ABS working principle b}\label{ABS working principle b}}
	\subfigure[]{\includegraphics[width=0.45\textwidth]{ABS working principle c}\label{ABS working principle c}}
	\subfigure[]{\includegraphics[width=0.45\textwidth]{ABS working principle d}\label{ABS working principle d}}
	\caption{ABS系统的工作原理}
	\label{ABS working principle}
\end{figure}

\subsection{以循环图示表述车载空调的工作原理, 并分析车载空调制冷与制热模式的区别。}

\cref{air conditioner}为车载空调开启制冷模式时的的循环图,其中蒸发器置于乘员舱内,冷凝器一般在车头的进气格栅后。流入蒸发器的制冷剂为低压液态,它在蒸发器中吸热为乘员舱降温,自身气化为低压蒸气。蒸发器后一般会设置鼓风机以通过对流加快热交换速度,从而增强制冷效果。低压气态的制冷剂沿管路流入压缩机,由发动机带动的压缩机为制冷剂加压使其变为高压气态,然后流入冷凝器中。由于压缩机对制冷剂的蒸气做功,高压蒸气的温度高于外界空气,在冷凝器中向大气放热冷凝为高压液态。同样地,冷凝器后也装有风扇,一般风扇的气流方向为吸气,外界空气经冷凝器流入风扇,加快冷凝器鳍片上的空气流动从而加快热交换。在冷凝器出口和膨胀阀间设有储液罐,起储存、干燥冷却液的功能,冷却液流过储液罐后仍为高压液态。膨胀阀有节流作用,高压液态制冷剂经过膨胀阀的节流孔节流后,变为低温低压的雾状,为制冷剂在蒸发器中的蒸发创造条件。同时膨胀阀还有控制制冷剂流量之功用。

车载空调制冷的原理在上一段已介绍,主要是通过制冷剂的相变和它与乘员舱内空气及外界空气的热交换实现,但空调制热的原理和制冷时区别很大。对于一般的发动机,它工作时本就会产生大量的热,需通过散热系统(\cref{engine cooling system})散出,我们可利用发动机产生的热量实现车载空调的制热。注意到在\cref{engine cooling system}中冷却液的管路上有着通往空调装置的进水口和出水口。由于冷却液温度一般维持在\SI{90}{\celsius}左右,高于一般人类需要的环境温度,我们把冷却液通入乘员舱内的空调暖风热交换器中并鼓风,就可在乘员舱中吹出热风了,而温度的控制也可通过控制流过热交换器的冷却液流量来实现。

当然,现在电动汽车的驱动电机发热量很小,不足以为车内供暖,甚至电池组还需专门的温度管理系统在低温时为电池加热防止容量的过大衰减。这时车载空调的制热仍需用到热泵或电热丝,显然这样的能耗是比较大的,为电动汽车本就在冬季不佳的续航表现雪上加霜。

\begin{figure}[htbp]
	\centering
	\begin{minipage}[b]{0.8\textwidth}
		\centering
		\includegraphics[width=\textwidth]{air conditioner}
		\caption{车载空调}
		\label{air conditioner}
	\end{minipage}
	\begin{minipage}[b]{\textwidth}
		\centering
		\includegraphics[width=0.8\textwidth]{engine cooling system}
		\caption{发动机冷却系统}
		\label{engine cooling system}
	\end{minipage}
\end{figure}

\subsection{电子车身稳定系统对车辆的意义,工作实现方式。}

车身电子稳定系统(ESP),是对旨在提升车辆的操控表现的同时、有效地防止汽车达到其动态极限时失控的系统或程序的通称。ESP是汽车防抱死制动系统(ABS)和牵引力控制系统(TCS)功能的进一步扩展,并在此基础上增加了车辆转向行驶时横摆率传感器、侧向加速度传感器和方向盘转角传感器,通过ECU控制前后、左右车轮的驱动力和制动力,确保车辆行驶的侧向稳定性。ESP是博世公司的专利产品,所以只有博世公司的车身电子稳定系统才可称之为ESP。

车身电子稳定系统的主要意义在于,确保在驾驶员操作欠妥或车辆极限工况时仍有较好的侧向稳定性,帮助车辆遵从驾驶者的转向意图,不会因为过度转向造成车辆甩尾,甚至翻车,也不会转向不足而无法紧急避开前方出现的障碍物。北美高速公路保险协会(IIHS)的研究表明,安装ESP能有效降低\SI{43}{\percent}致命的交通事故,美国高速公路安全管理局(NHTSA)进行的研究也表明,将ESP作为标准配置能有效降低\SI{34}{\percent}的轿车单车事故、\SI{71}{\percent}的轿车翻车事故,而SUV的单车事故甚至能降低\SI{59}{\percent}。

ESP主要由传感器、执行器和电子控制单元(ECU)三大部分组成(\cref{ESP components}),传感器一般包括轮速传感器、方向盘转角传感器、侧向加速度传感器、横摆角速度传感器、制动主缸压力传感器等,执行器一般包括传统制动系统、液压调节器等;ECU从传感器中读取信号,与内置策略比对后令执行器执行相应操作,也可与发动机管理系统联动,对发动机动力输出进行干预。

ESP系统的工作实现方式如下:

在一定的路面条件和车辆负载条件下,车轮能够提供的最大附着力为定值,即在极限情况下,车轮受到的纵向力(沿车轮滚动方向)与侧向力(垂直车轮滚动方向)为此消彼长的关系。更精确地,根据“摩擦圆”理论,轮胎在特定垂向载荷下能提供的侧向力和纵向力近似满足平方和为常数的关系,如\cref{friction circle}。ESP系统可分别控制各轮的纵向的制动力,间接地对侧向力施加影响,提高车辆的操控性能。

当纵向力达到极值,譬如车轮抱死时,车轮将失去侧向力附着力,车辆的横向运动会不受控制,发生侧滑,此时可能无法按司机的意愿进行转向。ESP系统可以检测并预防车辆侧滑。当ESP系统检测到车辆将要失控,它可向特定某个车轮施加制动力从而帮助车辆按照驾驶者期望的方向前进。

在转弯时,一种可行的控制策略为:当车辆有转向不足的倾向时,系统向转弯内侧的后轮增加制动力,利用差动转向的原理通过内外轮轮速差产生帮助转向的力矩;当有转向过度的倾向时,系统向转弯外侧的前轮施加制动力,由于此轮纵向力的增加,所能提供的侧向力减小,这样会产生抵抗转向的力矩,从而保证了行驶的稳定。部分与发动机管理系统联动的ESP系统还会在驱动轮打滑时,立刻减少节气门开度,降低发动机动力输出,同时对打滑的驱动轮轻微施加制动,这样就可以减少打滑并保持轮胎与地面间最佳附着,有利于在低附着系数的路面上的平稳起步。

\begin{figure}[htbp]
	\centering
	\begin{minipage}[b]{0.59\textwidth}
		\centering
		\includegraphics[width=\textwidth]{ESP components}
		\caption{ESP模块结构示意图}
		\label{ESP components}
	\end{minipage}
	\begin{minipage}[b]{0.4\textwidth}
		\centering
		\includegraphics[width=\textwidth]{friction circle}
		\caption{摩擦圆(右转时)}
		\label{friction circle}
	\end{minipage}
\end{figure}

\subsection{对于 ADAS 系统以及未来车辆电子化发展的感想。}

先进驾驶辅助系统(Advanced Driver Assistant System),简称ADAS,它利用安装于车上的各种传感器,在第一时间收集车内外的环境数据,进行静、动态物体的辨识、侦测与追踪等技术上的处理,让驾驶者在最快的时间察觉可能发生的危险,必要情况下可接管车辆以避免碰撞。ADAS系统能够有效提升驾驶安全性、舒适性,是实现L5级别自动驾驶前的过渡性技术。

ADAS(\cref{ADAS})的主要功能包括:自动泊车系统APA、自动巡航系统ACC、自动紧急刹车AEB、车道偏离预警系统LDW、车道保持系统LKA、前方碰撞预警FCW、行人碰撞预警PCW、车距监测警告HMW、交通标志识别TSR、远光灯辅助系统HBA等。

\begin{figure}[htbp]
	\centering
	\begin{minipage}[b]{\textwidth}
		\centering
		\includegraphics[width=\textwidth]{ADAS}
		\caption{ADAS系统}
		\label{ADAS}
	\end{minipage}
\end{figure}

从第一辆汽车诞生到现在已经过去百余年,一百年虽不长,但汽车已经对我们的社会生活和客观世界产生了深远的影响。实际上,汽车就和第二次工业革命的许多产物一样,为人类提高生产效率,解放生产力做出了巨大贡献。内燃机的发明使得动力装置能够被整合在一个更小的空间内,重量也大大降低,这是真正意义上的汽车出现的先决条件。第一次工业革命中实用的蒸汽机被发明,以蒸汽机为动力的火车和轮船成为第一次工业革命时期代表性的交通工具。第二次工业革命时有了内燃机,汽车和飞机也随即出现。以上四样交通工具可以说是承担了现代几乎所有的客货运输任务,承载着交通运输的命脉,其中又以汽车的使用最为广泛。

2020年全年中国汽车销量达到2531.1万辆,连续12年蝉联全球第一位。随着汽车在我国的普及,人们的出行获得了极大便利。不幸抑或又幸运的是,汽车行业在中国的飞速发展发生在21世纪初叶,虽有经历,年龄尚幼的我们似乎对此无甚切身感受。但我们可回顾历史,当初汽车被发明就是为替代马车,因为马车的速度受限,且不能长时间稳定地输出功率,长途旅行十分不易,无法满足人民日益增长的交通需求。乘马车从纽约到华盛顿有时需要一周,但若改用汽车,半天就能往返。汽车的出现改变了人们的出行习惯,人们可以去更远的地方旅游,甚至可以在远离市中心的地方落户,在不失去优渥岗位待遇的同时享受优美的环境和便捷的交通。

当然,汽车带给我们的绝非仅有便利。汽车的设计,受到种种工程上的制约,遍布着妥协的痕迹。随着 “新四化”的推进,相信未来汽车的发展方向将在于解决现阶段存在的种种问题。

据统计,2016年道路交通死亡人数达135万人,使其成为\numrange[range-phrase = $\,\sim\,$]{5}{29}岁儿童和青年人死亡的首要原因,以及所有年龄组死亡的第八大原因。每年因交通事故产生的直接经济损失占各国国民生产总值的\qtyrange[range-phrase = $\,\sim\,$, range-units = single]{1}{3}{\percent}。董幼鸿的一项研究指出,造成道路交通事故的最大因素在人。如果未来的汽车不再需要人驾驶,或者至少当车载系统发现危险时能主动介入,交通事故会减少吗?Eno Centre for Transportation给出的结论是肯定的。他们指出,如果\SI{90}{\percent}的美国道路上的汽车是自动驾驶的,那么事故的数量将从每年的600万下降到130万,死亡人数将从\num[group-separator={,}]{33000}下降到\num[group-separator={,}]{11300}。从实践上看,自动驾驶的安全性也没有让我们失望。2009年至2015年,谷歌无人驾驶汽车在300多万公里的行驶中仅发生了16起交通意外,且无一致命。当然,与特斯拉所谓“自动驾驶”系统有关的事故近年来常见诸报端,将白色货车车厢识别成天空导致追尾的若干起相似事故令人啼笑皆非。我们要清醒地认识到,图像识别等技术现在来看尚不是十分成熟,很多情况下没有人那么强大的环境适应能力,但在未来这个问题能不能得到解决呢?我认为是可以的。很多时候人不是没能力预见事故的发生,只是一时分心,或者经验不足。车规级的硬件可靠性能得到充分检验,不会因为连续工作而瞌睡走神。相关算法的改进一直在进行,即使是人也不能确保准确地识别所有危险,算法总有一天会在环境感知方面的能力超过人类,并像AlphaGo一样,越来越令人望尘莫及。

未来的汽车的电控系统只能起到辅助作用吗?如果仅起辅助作用,现在的ADAS系统已足够,但我认为L5级别的智能汽车迟早会大量上路,就像\cref{auto pilot}展示的那样,让开车变成坐车。不可避免地,不少人对激进的自动驾驶有抵触情绪,其中一个原因就是认为过高的自动化程度会剥夺驾驶人手动操作汽车的乐趣。徐逸伦等人认为,无人驾驶的开发满足了懒惰心理,使人类能力退化,并对驾培行业造成巨大冲击。对于智能汽车影响驾校收入这一点,我没有反对的理由,但其它的就有点耸人听闻了。就拿自动挡来说吧,自动挡汽车带来的“驾驶乐趣”、“操控感”显然不如手动挡,但驾驶自动挡要方便得多,厂家的调教也令自动挡汽车的燃油经济性比多数人手下的手动挡好,这些优势足以让手动挡轿车在市场的选择下沦为小众产品,除跑车外,很多车型虽然有手动挡的选择,但基本都是最低配,且很难买到。对于多数购车者来说,汽车最大的作用是作为一种交通工具被发挥的,我相信只要高阶的自动驾驶技术能够成熟并且价格亲民,路上的大部分汽车就会取消驾驶位。坐在没有司机的车上喝茶看报,让智能控制系统处理一切,将会成为很多人的日常。至于手动驾驶的爱好者呢?或许他们还可以继续开车上路,或许只能在封闭赛道上玩超跑了吧。电动车为了推广,有厂商在车身装设音响,模拟发动机的轰鸣,但现在这种行为越来越少。大部分人对技术创新乐见其成。

\begin{figure}[htbp]
	\centering
	\begin{minipage}[b]{0.6\textwidth}
		\centering
		\includegraphics[width=\textwidth]{auto pilot}
		\caption{自动驾驶场景设想}
		\label{auto pilot}
	\end{minipage}
\end{figure}

有人说,电动机是车用动力的未来,但我觉得车用动力完全可以多元化发展。电动车可以通过路面下预埋的线圈充电,但如果加氢站得到普及,氢能源动力同样方便。或许我们可以把动力系统当作汽车上一个可变更的模块,但无论动力为何,汽车网联化的发展都大有可期。现在汽车的决策都由车上一个个独立的驾驶员做出,就算有一些辅助驾驶,车辆之间也几乎没有沟通。设想在一条较为拥挤的公路上,某个司机不知什么原因,或许是看到了挡风玻璃前飘过的一个垃圾袋,或许是看到有只小狗在横穿马路,又或许只是稍微走了一下神,反正他重重地踩了一下刹车,车尾的刹车灯亮起,车辆明显减速。后车看到前车刹车,他也只能跟着刹,然后是后车的后车……如果大家车距都很近,那为了避免和前车相撞,整条路上的车都不得不先后减速,造成道路通行能力下降,即使第一辆车遇到的危险早已排除。当然,我们可以控制车距,只要车距足够让第一辆车解决完问题,后面的所有车就都不会受到影响,但我觉得这种办法就像某些人说的通过保持\SI{50}{meter}以上的车距以避免在大货车后看不到交通信号灯的问题一样,实属正确而无用。但如果每辆车之间都能沟通,前述问题就很容易得到解决。前车刚要刹车,后面的所有车就都能得知这一情况,同时刹车;危险排除,前车加速时,后车同步加速。省去了反应时间,减速加速的循环就不会像波一样逐个地向后传递。保持适当的车距,还可通过牺牲部分车距来换取刹车加速度的减少,类似缓冲一样,让刹车、加速的过程只影响到有限的几辆车。当然,上面只是网联化在一个很具体的场景下体现出的优势,似乎显得过于理想化。如果汽车能和路面,和周围的交通摄像头互联呢?或许汽车在情况发生之前就可以反应,无需人工干预。汽车的决策系统在分析同僚的数据后做出一系列优化,乘客只会觉得车辆的行驶异常平稳。

未来汽车看起来会是怎样?高度电子化的汽车看起来应该也和现在大不相同吧。现在车身的设计,为了兼顾视野和安全性,人们坐进车内最直观看到的就是前挡、侧窗玻璃,以及玻璃旁的横梁、纵梁、ABC柱等(\cref{car-body structure})。车身设计的主要目标,包括降低风阻、提高乘员安全性等。汽车上任何部件的设计都是妥协的艺术,有车人士津津乐道的盲区,就是由各种立柱造成的。今年新款的轿车和十年前的在外观上看起来有很大变化吗?\cref{Audi A6 2012}所示的车发布于2012年,但和今年的新车有什么区别?专业的汽车设计师当然会说有,每一毫米曲线的调整都是为了追求将更完美的细节呈现在大家面前。但实事求是地说,技术没有革命性的变化,外观自然也就不会大变。

\begin{figure}[htbp]
	\centering
	\begin{minipage}[b]{0.45\textwidth}
		\centering
		\includegraphics[width=\textwidth]{car-body structure}
		\caption{车体结构图}
		\label{car-body structure}
	\end{minipage}
	\begin{minipage}[b]{0.5\textwidth}
		\centering
		\includegraphics[width=\textwidth]{Audi A6 2012}
		\caption{Audi A6 2012}
		\label{Audi A6 2012}
	\end{minipage}
\end{figure}

然而,车身也不可能始终一成不变。立柱是为了保证车身强度,增强碰撞安全性,但如果未来主动安全的技术充分发展,汽车发生碰撞事故的可能性变得微乎其微,立柱还有存在的必要吗?事实上,客机的设计就走过类似的道路。过去发生的几起惨烈的撞机事件,促使工程师们思考解决对策。最显而易见的方法就是加强机身,令其在撞击后仍能保证乘员安全空间,但这在当时的技术条件下无法实现。于是他们另辟蹊径,研发出TCAS系统,让所有大飞机间实时相互通报位置,极大地降低了空中撞击风险,而机身仍可保持轻巧。因此,随着汽车主动安全技术的发展,被动安全的要求可能会逐步降低。如果有一天,钢化玻璃就能满足碰撞要求,可以期待会有车企造出全透明的车身,给驾乘人员带来最好的视野。

或者我们可以换一个思路,将立柱隐藏起来?在一些先进战机,如美国的F-35上,飞行员的头盔显示器连接到机外的分布式孔径系统,就算是飞机下方的景象都能一览无余。在汽车上也是同样地,把装设在车外相机拍摄到的图像实时投影在车内非透明部分上,就能达到类似“透明”的效果。在目前的技术下,这种方案可能存在成本高、显示效果不逼真、延迟大等问题,和玻璃一点都不像,未在任何一款量产车上得到应用,但这些问题不是根本性的,实时投影车外图像的方式未来可期。

本茨造的第一辆汽车,不过是把马车的马具去掉,在后轮上方放置发动机和传动装置(\cref{the first car}),这也符合当时对车的基本认识。但经过百年的优化,汽车变得和马车一点都不像。常见的三厢车由发动机室、驾驶室、行李箱组成,可以认为,这种布局是比较适合以内燃机为动力的轿车的。很多人认为电动汽车是未来,但电动车的电池包装在车底,电动机也在底盘上,甚至于未来的轮边电机、轮毂电机,那发动机室的意义何在?可以预见的是,电动汽车的大范围铺开应当会给车身的布局带来较大的变化,但现在就算是特斯拉这种新兴势力造出的电动车基本上还是沿用燃油车的布局(\cref{TESLA Model S})。未来在没有动力装置约束的情况下,或许汽车的曲线会变得更平滑?或许乘员空间、储物空间会变大?可能真正革命性的改变就像乔布斯的智能手机一样,发布之前无法想象,但大家上手用了都能体会到设计之精巧吧。

\begin{figure}[htbp]
	\begin{minipage}[b]{0.4\textwidth}
		\centering
		\includegraphics[width=\textwidth]{the first car}
		\caption{第一辆汽车(仿制品)}
		\label{the first car}
	\end{minipage}
	\begin{minipage}[b]{0.55\textwidth}
		\centering
		\includegraphics[width=\textwidth]{TESLA Model S}
		\caption{TESLA Model S}
		\label{TESLA Model S}
	\end{minipage}
\end{figure}

\clearpage

\section{底盘部分}
\subsection{麦弗逊悬架中的减震器的轴线为什么和螺旋弹簧的轴线不重合?}
\subsection{多连杆悬架那些四轮定位定位参数可以调整,如何调整?}
\subsection{试叙述减振器的工作原理和几个功能阀的作用。}
\subsection{你拆有哪些类型的转向器?各有什么特点?如何调整转向系统间隙?}
\subsection{摩擦式离合器自由间隙的位置,并说出为什么要有自由间隙。自由间隙的过大过小有什
	么不利的影响? 车辆在使用过程中自由间隙是怎样变化的?}
\subsection{试述液力变矩器工作原理,其泵轮、导轮和涡轮分别与什么部件相连接?}
\subsection{简述鼓式制动器间隙自动调整原理。}
\subsection{简述真空助力泵的工作原理。}
\subsection{简述万向节的构造特性、种类以及速度特性。}
\clearpage


\end{document}